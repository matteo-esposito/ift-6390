%%%%%%%%%%%%%%%%%%%%%%%%%%%%%%%%%%%%%%%%%%%%%%%%%%%%%%%%%%%%%
\section{\enfr{Practical Part}{Partie pratique} \points{7}{7}}
\instruct{%
\enfr{You should work off of the template file \texttt{solution.py} in the project and fill in the basic numpy functions using numpy and python methods}%
{Vous devez travailler sur le modèle \texttt{solution.py} du répertoire et compléter les fonctions basiques suivantes en utilisant numpy et python}}

\begin{enumerate}

\item \points{1}{1}
\enfr{Create a numpy array from a python list}%
{Crée un tableau numpy à partir d'une liste python}

\item \points{1}{1}
\enfr{Create a numpy array of length 1 from a python number}%
{Crée un tableau numpy de taille 1 à partir d'un nombre python}

\item \points{1}{1}
\enfr{Sum two arrays elementwise}%
{Additionne deux tableaux élément par élément}


\item \points{1}{1}
\enfr{Sum an array and a number}%
{Additionne un tableau et un nombre}

\item \points{1}{1}
\enfr{Mutliple two arrays elementwise}%
{Multiplie deux tableaux élément par élément}


\item \points{1}{1}
\enfr{Dot product of two arrays}%
{Calcule le produit scalaire euclidien (dot product) de deux tableaux}

\item \points{1}{1}
\enfr{Dot product of an array (1D) and a matrix (2D array)}%
{Calcule le produit scalaire euclidien (dot product) d'un tableau (1D) et d'une matrice (tableau 2D)}

\end{enumerate}
