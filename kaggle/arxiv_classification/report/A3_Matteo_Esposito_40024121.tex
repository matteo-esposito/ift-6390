\documentclass[a4paper,12pt]{article}

% Packages
\usepackage{float}
\usepackage[utf8]{inputenc}
\usepackage{fancyhdr}
\usepackage[left = 0.6 in, right = 0.6 in, top = 1 in, bottom = 1 in, headsep = 0.5 in]{geometry}
\usepackage[normalem]{ulem}
\usepackage{enumerate}
\usepackage{pgfplots}
\usepackage{titling}
\pgfplotsset{width=8cm,compat=1.9}
\usepackage{stackengine}
\usepackage{algorithm}
\usepackage[noend]{algpseudocode}
%\pagenumbering{gobble} 
\usepackage{amsmath} % add [fleqn] before {amsmath} to left center
\usepackage{amssymb}
\usepackage[T1]{fontenc}
\usepackage{fancybox}
\usepackage{longtable}


\hfuzz = 100pt

%%% Horizontal Line Command
\makeatletter
  \newcommand*\variableheghtrulefill[1][.4\p@]{%
    \leavevmode
    \leaders \hrule \@height #1\relax \hfill
    \null
  }
\makeatother
%%%

%%% Algorithm commands
\algblock{Input}{EndInput}
\algnotext{EndInput}
\algblock{Output}{EndOutput}
\algnotext{EndOutput}
\newcommand{\Desc}[2]{\State \makebox[3em][l]{#1}#2}
%%%

%%% Base conversions
% expandable loop (used to avoid scope problems in tabular cells with the
% standard \loop)
\def\boucle #1\repeat {#1\b@@cle {#1}\repeat \repeat }
\def\b@@cle #1{\repeat #1\b@@cle {#1}}

\makeatletter
\newcount\@nn
\newcount\@mm
\newcount\@base
\newcount\@baseminusone

% please do not use this at home
% #1 must be a counter name, not something expanding to a number.
\def\@arabalpha #1{\ifcase #10\or1\or2\or3\or4\or5\or6\or7\or8\or9\or 
A\or B\or C\or D\or E\or F\or G\or H\or I\or J\or K\or L\or M\or N\or O\or
P\or Q\or R\or S\or T\or U\or V\or W\or X\or Y\or Z\fi}

\newcommand{\baseexpansion}[2][2]{% no negative numbers please!
\def\@digits{}%
\@base#1\relax \@baseminusone\@base\advance\@baseminusone-1
\@nn #2\relax  % this is the number to be written in base #1
%
\ifnum\@baseminusone<36
\def\onerow{#1\kern.1em\hbox{\vrule
   \vtop {\hbox{\ \the\@nn}\kern.3ex\hrule height.1ex }} &%
   \global\@mm\@nn \global\divide\@mm\@base 
   \multiply\@mm\@base \advance\@nn-\@mm 
   \the\@nn \xdef\@digits{\@arabalpha\@nn\@digits}}%
\else
\def\onerow{#1\kern.1em\hbox{\vrule
   \vtop {\hbox{\ \the\@nn}\kern.3ex\hrule height.1ex }} &%
   \global\@mm\@nn \global\divide\@mm\@base 
   \multiply\@mm\@base \advance\@nn-\@mm 
   \the\@nn \xdef\@digits{\the\@nn.\@digits}}%
\fi
%
\leavevmode\oalign{$#2_{10}:$\hfil\cr
      $\left.
      \begin{tabular}{r|l}
         \boucle \onerow \\ \ifnum\@nn>\@baseminusone\global\@nn\@mm \repeat
      \end{tabular}\right\rbrace=
      \mathtt{\@digits}_{#1}$}}     % \hfil removed from the macro

\makeatother

%%% Logic Circuits
\usetikzlibrary{arrows, shapes.gates.logic.US, calc}
\tikzstyle{branch}=[fill, shape=circle, minimum size=3pt, inner sep=0pt]

%%% Header & Footer
\fancyhf{}
\renewcommand{\footrulewidth}{0.1mm}
\fancyhead[L]{Matteo Esposito}
\fancyhead[R]{COMP 228-AA}
\fancyfoot[R]{\thepage}
\pagestyle{fancy}

%%% Code formatting
\def\code#1{\texttt{#1}}
\renewcommand\maketitlehooka{\null\mbox{}\vfill}
\renewcommand\maketitlehookd{\vfill\null}
\setlength{\abovedisplayskip}{2pt}
\setlength{\belowdisplayskip}{3pt}

%%% Custom settings
\setlength{\parskip}{0.75em}  % Paragraph spacing
\setcounter{section}{-1} % Page numbers to start on page 1
\setlength\parindent{0pt} % Remove indenting from entire file
\def\layersep{2.5cm}

%%% Titlepage
\title{\textbf{Assignment 3}}
\author{COMP 228-AA | System Hardware}
\date{Matteo Esposito (40024121)}

%--------------------------------------------------------------------------%

\begin{document}

\begin{titlingpage}
  \maketitle
  \centering
  \vfill
  {\large{Concordia University}}\par
  {\large{June 13, 2019}}
\end{titlingpage}

\newpage

\subsection*{Question 1}

\textbf{a)} Since $2^6 = 64$ and $2^7 = 128$, if we are dealing with 87 unique instructions it is clear that we need 7 out of the 24 available bits from the standard word size to represent our instruction/opcode set.

\textbf{b)} Since we have 17 remaining bits for memory addresses, the maximum value that can be represented in binary is $2^{18}-1 = 262143_{10}$.

\textbf{c)} Knowing that the range of N-bit 2's complement values is $[-2^{n-1}, 2^{n-1}-1]$, and considering that we have 24 bit words, the range and max of 2's complement values is: 

$$[-2^{n-1}, 2^{n-1}-1] = [-2^{23}, 2^{23}-1] = [-8388608, 8388607] \Rightarrow \fbox{8388607}$$

\subsection*{Question 2}

Interpreting each set of 4 bits from the Marie snippet \code{121830F46000900A}, where bit 1 represents the instruction number and bits 2-4 the memory address, we get the 4 following instructions:

\begin{table}[H]
    \centering
    \begin{tabular}{|c|c|c|c|}
        \hline
        \text{\textbf{Hex}} & \text{\textbf{Opcode}} & \text{\textbf{Address}} & \text{\textbf{Instruction}} \\ \hline
        1218 & 1 = \code{LOAD} & $218_{16}$ & \code{LOAD 218} \\ \hline
        30F4 & 3 = \code{ADD} & $0F4_{16}$ & \code{ADD 0F4} \\ \hline
        6000 & 6 = \code{OUTPUT} & - & \code{OUTPUT} \\ \hline
        900A & 9 = \code{JUMP} & $00A_{16}$ & \code{JUMP 00A} \\ \hline
    \end{tabular}
\end{table}

\subsection*{Question 3}

\begin{figure}[H]
    \centering
    \caption{Timing digram for \code{LOADI} instruction}
    \includegraphics[width=12cm]{figures/q3.jpg}    
\end{figure}

\subsection*{Question 4}

\textbf{a)} Converting every bit in $AAFB1609_{16}$ to binary, we get,

$$AAFB1609_{16} = 10101010111110110001011000001001_{2}$$

Converting this to two's complement notation (flip all bits and add 1) we get,

$$10101010111110110001011000001001_{2} \rightarrow 
-(01010101000001001110100111110110_{2})$$

then,

$$-(01010101000001001110100111110110_{2}) + 1 =
-(01010101000001001110100111110111_{2})$$

converting back to decimal,

$$-(01010101000001001110100111110111_{2}) = -1426385399$$

\textbf{b)} Converting every bit in $0916FBAA_{16}$ to binary, we get,

$$0916FBAA_{16} = 00001001000101101111101110101010_{2}$$

Using the IEEE standard for single precision floating point numbers, namely to use bit 1 as the sign bit, the next 8 for the exponent and remaining 23 for the mantissa we get,

$$0|00010010|00101101111101110101010$$

This yields an exponent value of $00010010_2 = 18_{10}$. If we inversely apply the bias to get the original value of the exponent, we have 18 - 127 = -109. If we convert the mantissa, we get $1.00101101111101110101010_{2} \; x \; 2^{-109}$. We will translate the solution by 1 power to the right. $0.1001 0110 1111 1011 1010 1010_{2} \; x \; 2^{-108}$. Converting to hex then decimal we get the following:

\begin{align*}
    0.96FBAA_{16} \; x \; 2^{-108} &= (9\cdot16^{-1} + 6\cdot16^{-2} + 15\cdot16^{-3} + 11\cdot16^{-4} + 10\cdot16^{-5} + 10\cdot16^{-6}) \; x \; 2^{-108} \\
    &= 0.589777589 \; x \; 2^{-108}        
\end{align*}

Converting $2^{-108}$ to an exponent of 10, we get:

\begin{align*}
    2^{-108} &= 10^{x} \\
    -108 \cdot log_2(2) &= x\cdot log_2(10) \\
    \frac{-108}{log_2(10)} &= x \\
    -32.5112395 &\approx x
\end{align*}

finally,

$$0.589777589 \; x \; 2^{-108} = 0.589777589 \; x \; 10^{-32.5112395}$$

To get a whole number exponent, we can multiply the coefficient by a constant.

$$\frac{0.589777589}{10^{-33+32.5112395}} \; x \; (10^{-32.5112395})*(10^{-33+32.5112395}) = 1.8173926 \; x \; 10^{-33}$$

\subsection*{Question 5}

$$Q = (A+B)/C - D*E$$

\textbf{a)} 

\begin{align*}
    &\code{Push A} \\
    &\code{Push B} \\
    &\code{add} \\
    &\code{Push C} \\
    &\code{Div} \\
    &\code{Push D} \\
    &\code{Push E} \\
    &\code{Mult} \\
    &\code{Sub} \\
\end{align*}

Following the \code{Sub} call, we need to store/pop the final result (Q) from the stack.

\textbf{b)}

\begin{align*}
    &\code{Add R1, A, B} \\
    &\code{Div R2, R1, C} \\
    &\code{Mult R3, D, E} \\
    &\code{Sub R4, R2, R3} \\
\end{align*}

Q will be stored in register R4.

\subsection*{Question 6}

Calling \code{LOAD 500}.

\begin{figure}[H]
    \centering
    \begin{tabular}{|c|c|}
        \hline
        \textbf{Mode} & \textbf{Value Loaded into AC}  \\ \hline
        \text{Indirect} & 100\\ \hline
        \text{Indexed} & 300\\ \hline
        \text{Direct} & 300\\ \hline
        \text{Immediate} & 500\\ \hline
    \end{tabular}
\end{figure}

\subsection*{Question 7}

See \code{mult.mas}.
\par
\underline{Note:} All inputs must be entered in the ASCII (default) input setting for correctly formatted output.

\end{document}